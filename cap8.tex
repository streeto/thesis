%%%%%%%%%%%%%%%%%%%%%%%%%%%%%%%%%%%%%%%%%%%%%%%%%%%%%%%%%%%%%%%%%
% Qualificacao de Doutorado / Dept Fisica, CFM, UFSC            %
% Andre@UFSC - 2015                                             %
%%%%%%%%%%%%%%%%%%%%%%%%%%%%%%%%%%%%%%%%%%%%%%%%%%%%%%%%%%%%%%%%%


%:::::::::::::::::::::::::::::::::::::::::::::::::::::::::::::::%
%                                                               %
%                          Capítulo 9                           %
%                                                               %
%:::::::::::::::::::::::::::::::::::::::::::::::::::::::::::::::%

%***************************************************************%
%                                                               %
%                           Conclusao                           %
%                                                               %
%***************************************************************%

\chapter{Conclusões e perspectivas}
\label{sec:conclusao}

O {\em survey} CALIFA observou cerca de 600 galáxias, nas configurações V500 e
V1200. Foram publicados cubos de 100 galáxias no primeiro {\em data release}, e
de 200 galáxias no segundo. Os cubos das galáxias restantes ainda estão em
processo de controle de qualidade e embargo da colaboração. Através do programa
\starlight, foram obtidos diversos parâmetros relacionados às populações
estelares componentes de cada {\em spaxel}, formando cubos de dados adicionais
ao observados, com a as propriedades físicas espacialmente resolvidas obtidas a
partir da saída do \starlight.

Neste trabalho, foi desenvolvido um programa chamado PyCASSO para organizar e
analisar estes cubos de dados. Este programa é utilizado por cerca de 10 pessoas
que estudam populações estelares na colaboração do CALIFA. Bastante atenção foi
dada à documentação (Seção \ref{sec:pycasso:Pycasso}, com o manual completo no
Apêndice \ref{apendice:manual}). Foram publicados 4 artigos que se baseiam
fortemente no uso de PyCASSO para análise e gráficos, resumidos na Seção
\ref{sec:pycasso:art} e presentes como apêndices deste trabalho (Apêndices
\ref{apendice:PaperResolving1}, \ref{apendice:PaperResolving2},
\ref{apendice:InsideOut} e \ref{apendice:RadStruct}).

PyCASSO já foi usado com sucesso com dados de outros {\em surveys} como o PINGS
\citep{RosalesOrtega2010}, um precursor do CALIFA, porém com até $10$ vezes mais
espectros por galáxia. MaNGA \citep{Bundy2015} está em andamento, e deverá obter
IFS de $10.000$ galáxias. Muito embora tenha um número muito maior de galáxias
do que o CALIFA, sua cobertura espacial será menor. Assim, estes {\em survey}
deverá ser complementar ao CALIFA. Testes com dados preliminares do MaNGA foram feitos
com PyCASSO, requerendo apenas pequenas modificações para que funcionem
normalmente.

É importante que PyCASSO se torne um programa modular, onde o usuário cria uma
descrição do arquivo de IFS de forma que o programa saiba como ler os dados.
Recentemente o autor deste trabalho, junto com pesquisadores do Grupo de
Astrofísica da UFSC, começou a colaborar com um novo {\em survey} chamado
Diving3D. Este {\em survey}, descrito em \citet{Ricci2014}, vai obter IFS de
quase 200 núcleos de galáxias utilizando o Telescópio Gemini. Os cubos de dados
espectrais são de uma qualidade excepcional, e já iniciamos o a adaptação de
ferramentas como o PyCASSO e o \starlight para trabalhar com esses dados. O
objetivo, do ponto de vista ferramental, é que se crie uma {\em pipeline} onde
dados como populações estelares, morfologia, etc, vão sendo adicionados de forma
incremental aos cubos, mantendo uma interface comum para análise dos dados.

Foi desenvolvida uma técnica de síntese espectral aplicada às componentes
morfológicas (bojo e disco) de galáxias. A decomposição morfológica do perfil de
brilho em uma imagem de uma galáxia é um campo de estudo bem desenvolvido, com
ferramentas bastante eficientes disponíveis na comunidade acadêmica (GALFIT,
BUDDA e IMFIT, por exemplo). A ideia aqui foi se valer destas ferramentas, junto
com PyCASSO, e realizar a decomposição para imagens em cada comprimento de onda
dos cubos de dados espectrais.

Ao fazer a decomposição morfológica em janelas estreitas de comprimento de onda,
os efeitos de cinemática precisaram ser levados em consideração. Em uma imagem
de uma galáxia, em um determinado comprimento de onda, não se observa os mesmos
processos físicos em todos os {\em pixels}. A luz de cada {\em pixel} é composta
pela luz de diversas estrelas, cada uma com uma velocidade diferente em relação
à linha de visada (e consequentemente um {\em redshift} ou {\em blueshift}
diferente), com cada {\em pixel} contendo um conjunto distinto destas
velocidades. Para mitigar este problema, a distribuição de velocidades de cada
{\em spaxel} dos cubos de dados foi determinada, de forma aproximada, por uma
velocidade sistêmica e uma dispersão de velocidades. Os {\em spaxels} foram
colocados todos no {\em rest frame}, e com a mesma dispersão de velocidade (ver
Seção \ref{sec:Decomp:cinematica} para detalhes). Isto fez com que a resolução
fosse degradada, mas garante que se observa a luz da galáxia inteira de forma
consistente.

A PSF é uma peça chave neste quebra-cabeças morfológico. Sem uma boa medida da
forma da PSF, a decomposição morfológica pode não funcionar, ou pior, gerar
resultados equivocados. No Capítulo \ref{sec:psf}, foi descrito como  a PSF do
CALIFA foi medida, um resultado que acabou sendo adotado pela colaboração,
publicado no segundo {\em data release} \citep{GarciaBenito2015}. A PSF obtida
segue um perfil de Moffat, com $\beta = 4$ e $\mathrm{FWHM}=2,9 \pm 0,3\,\arcs$.
Com esta incerteza na estimativa da largura da PSF, testes foram feitos para
determinar o efeito que um erro na PSF causa no resultado da decomposição
morfológica, apresentados na Seção \ref{sec:test:psf}. Em via de regra, a
decomposição dos espectros em componentes morfológicas, quando se utiliza a PSF
errada, apresenta artefatos como gradientes de cores e linhas espectrais
artificiais. Estes problemas parecem ser atenuados quando se utiliza os
espectros integrados das componentes morfológicas.

No Capítulo \ref{sec:Decomp} foi escolhida, dentre as galáxias do CALIFA
observadas com a configuração V500, uma amostra de 43 galáxias que foram
classificadas com boa possibilidade de serem da classe S0 (listada no Apêndice
\ref{apendice:amostra}). O objetivo era ter galáxias que pudessem ser bem
ajustadas a um modelo bojo--disco. Destas 43, 19 tinham características que não
permitiam o uso deste modelo, seja por faixas de poeira, uma das componentes
muito fraca, ou problema nos dados, entre outras coisas. Outras 15 tiveram
problema no ajuste do modelo inicial. Somente 9 galáxias tiveram um relativo
sucesso na decomposição morfológica espectral. O resultado da decomposição para
todas as 9 galáxias da amostra final é mostrado nas figuras do Apêndice
\ref{apendice:Decomp}. A galáxia K0858, classificada como S0a, foi escolhida
para uma análise mais detalhada, em parte pela sua decomposição morfológica ter
sido muito boa, e em parte por ter um bom ajuste com o \starlight.

Em K0858 verificou-se que há um gradiente espectral em alguns dos parâmetros
morfológicos, que foram ajustados levando em conta apenas informações espaciais.
Entretanto, não se pode descartar que estes gradientes, e outras características
menores observadas na dependência dos parâmetros com o comprimento de onda, não
sejam artefatos da forma como a decomposição foi feita. O teste com os erros na
PSF deixaram isto bem claro, assim como os resíduos nos espectros (vide Figura
\ref{fig:decompSpectra}). É preciso estudar a forma como os parâmetros
morfológicos deveriam variar com o comprimento de onda. Compreender a forma como
as populações estelares se distribuem nas componentes morfológicas de uma
galáxia, e como estas populações se refletem nos espectro espacialmente
resolvido da galáxia, é fundamental para poder fazer uma decomposição
morfológica que leve em conta informações das regiões vizinhas no espectro.

A síntese de populações estelares foi realizada com sucesso apenas em K0592
(classificada como E0), e K0858. O ajuste dos espectros e as propriedades
físicas, para todas as 9 galáxias da amostra, estão disponíveis no Apêndice
\ref{apendice:Decomp}. Apenas K0592 e K0858 tiveram um resíduo no ajuste do
\starlight pequeno o bastante para poderem ter as propriedades físicas obtidas
para seus bojos e discos levadas a sério. A Figura \ref{fig:decompSintese}
mostra idade e metalicidade estelar médias para as componentes das duas
galáxias, calculados sobre os espectros integrados. Esta figura indica que os
bojos apresentam populações mais velhas e menos metálicas, enquanto os discos
apresentam populações mais jovens e mais metálicas, em relação ao espectro
observado.

Um estudo que poderia ser feito com os resultados da síntese de populações
aplicados à decomposição morfológica espectral é medir a densidade superficial
de massa estelar do disco, que fica normalmente ofuscado pelo bojo nas regiões
centrais. O objetivo é verificar previsão de \citet{Freeman1970}, que diz que a
densidade superficial de massa do núcleo dos discos de galáxias é sempre o
mesmo, independente do tipo morfológico e do perfil de massa que a galáxia
tenha. Todavia, a síntese de populações espacialmente resolvida não é muito
robusta da forma como é feita aqui, como visto na Figura
\ref{fig:decompSinteseRadprof}. Entre as prováveis causas está a degenerescência
no modelo bojo--disco, com 7 parâmetros livres. Deixar parâmetros como P.A. e
$\epsilon$ fixos pode ajudar, mas isso requer testes.

Muitos problemas podem ter sido causados pela forma como se trata a cinemática.
Talvez seja preciso fazer testes com modelos dinâmicos simples de galáxias para
verificar se a abordagem utilizada é mesmo adequada. Outra alternativa seria
fazer a decomposição de forma iterativa. Depois de ter determinado as
componentes morfológicas, utilizar este conhecimento para tentar medir a
cinemática das duas componentes nos dados originais, e tentar fazer uma
correção melhor, repetindo a decomposição. Não é óbvio, entretanto, que esta
abordagem vá realmente convergir, ou mesmo se é factível computacionalmente.

Mesmo com todos estes problemas em potencial, o passo lógico seguinte é tentar
aplicar o ajuste a galáxias espirais. A própria K0858 apresenta um resíduo no
ajuste morfológico que parece um braço espiral (ver Figura
\ref{fig:decompImages:K0858}). Os braços espirais têm um papel importante na
formação estelar, e com esta abordagem pode-se tentar obter o histórico de
formação estelar nestas regiões. Também pode-se tentar utilizar um outro modelo
morfológico para ajustar galáxias com um grande ângulo de inclinação. Com mais
galáxias, seria possível aplicar o estudo a uma boa fração da amostra do CALIFA,
e obter uma melhor estatística.

% End of this chapter
