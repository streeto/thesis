%%%%%%%%%%%%%%%%%%%%%%%%%%%%%%%%%%%%%%%%%%%%%%%%%%%%%%%%%%%%%%%%%
% Dissertacao de Mestrado / Dept Fisica, CFM, UFSC              %
% Andre@UFSC - 2011                                             %
%%%%%%%%%%%%%%%%%%%%%%%%%%%%%%%%%%%%%%%%%%%%%%%%%%%%%%%%%%%%%%%%%

%***************************************************************%
%                                                               %
%                          Resumo                               %
%                                                               %
%***************************************************************%

\begin{abstract}[Resumo]

Neste trabalho foram desenvolvidas ferramentas para trabalhar com espectros de
campo integral (IFS) do {\em survey} CALIFA. Os espectros dos {\em spaxels} são
preprocessados e em seguida analisados com o uso do programa \starlight. Uma das
ferramentas principais discutidas aqui, PyCASSO, organiza os arquivos de saída
do \starlight em cubos de dados de produtos da síntese de população estelar. Ele
também permite uma programação interativa e exploratória, dando acesso de forma
prática e simples aos dados multidimensionais.

Através do uso destas ferramentas, foi desenvolvido um método que obtém e
analisa as populações estelares das componentes morfológicas (bojo e disco) de
galáxias, a partir de dados de IFS. A decomposição morfológica é feita
utilizando o programa IMFIT, com um {\em wrapper} em Python. Uma amostra de 43
galáxias classificadas com S0 e com baixa inclinação foi escolhida para
aplicação do método. O modelo morfológico utilizado foi um bojo com perfil de
Sérsic e um disco exponencial. A decomposição morfológica é feita a cada
comprimento de onda, de tal forma que se obtém ao final um espectro para cada
{\em pixel} do bojo e do disco. Uma boa medida da PSF é essencial neste
procedimento, então foi feita a caracterização da PSF do CALIFA utilizando as
estrelas de calibração do {\em survey}. Os parâmetros morfológicos ($r_e$, $n$,
P.A. e $\epsilon$ para o bojo, $h$, P.A. e $\epsilon$ para o disco), na maioria
dos casos, depende linearmente, em média, do comprimento de onda, mas o seu
comportamento a cada $\lambda$ ainda não é bem compreendido. Foi feita a síntese
espectral de populações estelares das componentes morfológicas de 9 galáxias da
amostra, que tiveram um bom resultado na decomposição. Apenas duas destas
produziram bons ajustes com o \starlight, CALIFA 0592 (NGC 4874) e CALIFA 0858
(UGC 10905). Em ambos os casos se obtém um bojo mais velho e menos metálico e um
disco mais jovem e mais metálico do que o resultado da síntese do espectro
observado. A síntese de populações estelares utilizando os espectros integrados
produziram resultados mais robustos. Os espectros espacialmente resolvidos do
bojo e do disco parecem ter artefatos que interferem no ajuste do \starlight,
sendo interpretados como poeira, entre outras coisas, levando a resultados
equivocados.

\clearpage

% Ou\ldots
% 
% a figura a seguir resume esta tese.
% 
% \begin{center}
% \includegraphics[width=0.7\textwidth]{figuras/lmp}
% \end{center}

\end{abstract}

% End of resumo
