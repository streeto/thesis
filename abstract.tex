%%%%%%%%%%%%%%%%%%%%%%%%%%%%%%%%%%%%%%%%%%%%%%%%%%%%%%%%%%%%%%%%%
% Dissertacao de Mestrado / Dept Fisica, CFM, UFSC              %
% Andre@UFSC - 2011                                             %
%%%%%%%%%%%%%%%%%%%%%%%%%%%%%%%%%%%%%%%%%%%%%%%%%%%%%%%%%%%%%%%%%

%***************************************************************%
%                                                               %
%                          Abstract                             %
%                                                               %
%***************************************************************%

\begin{abstract}

We developed a method for recovering and analysing the stellar populations of
the different morphological components of galaxies using integral spectroscopy
data. Using the software Imfit, wrapped in Python, we perform the decomposition
of a sample of 53 candidate S0 galaxies from the CALIFA Survey into a Sérsic
bulge and an exponential disk. The decomposition is made wavelength-wise, so
that at the end we get the bulge and disk spectra for each pixel. The
morphological parameters ($R_e$, $n$, P.A. and $\epsilon$ for bulge, $h$, P.A.
and $\epsilon$ for disk) in most cases depends linearly on the wavelength, at
least for wavelengths $> 4000\,\angstrom$. Using the decomposed spectra from two
galaxies from the sample (NGC 1375 and NGC 7194), we apply a stellar population
synthesis using STARLIGHT, in order to recover the stellar populations of bulge
and disk. In both cases we recover an old bulge and an intermediate aged disk.
We also obtain a weak but positive stellar age gradient for the bulges and no
measurable gradient for the disks.


\end{abstract}

% End of abstract



