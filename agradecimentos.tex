%%%%%%%%%%%%%%%%%%%%%%%%%%%%%%%%%%%%%%%%%%%%%%%%%%%%%%%%%%%%%%%%%
% Dissertacao de Mestrado / Dept Fisica, CFM, UFSC              %
% Andre@UFSC - 2011                                             %
%%%%%%%%%%%%%%%%%%%%%%%%%%%%%%%%%%%%%%%%%%%%%%%%%%%%%%%%%%%%%%%%%


%***************************************************************%
%                                                               %
%                        Agradecimentos                         %
%                                                               %
%***************************************************************%

\chapter*{Agradecimentos}

Agradeço primeiro à minha esposa Marilize, que tanto me apoiou nesta aventura, e
com quem terei junto a maior aventura de todas, que vai chegar logo, logo. 
Agradeço também aos meus pais e irmãos pelo amor e companheirismo, e por eu
sempre poder contar com eles.

Ao meu orientador e grande amigo Cid, por todas as oportunidades, e pelas
infindáveis discussões, às vezes sobre esta tese, e muitas vezes muito bem
regadas. À Rosa, minha orientadora no IAA, que nos recebeu, eu e Marilize, com
tanto carinho tão longe de casa. E principalmente por dar um norte científico a
alguém que costumava pensar primeiro nos aspectos técnicos do trabalho.

Também ao Enrique, à Natalia e ao Rubén, sempre com ideias geniais, muitas das
quais encontrando o seu caminho para esta tese. Se há resultados bons nestes
trabalho, seus conselhos, dicas e sugestões tiveram um papel fundamental.

Aos colegas de estudos, William, Eduardo, Fábio, Rafa, Clara e todo o pessoal
do Grupo de Astrofísica e do IAA, pelo ambiente agradável de trabalho; muitas
amizades que devo levar pro resto da vida.

Este trabalho não seria possível sem os recursos financeiros da Coordenação de
Aperfeiçoamento de Pessoal do Nível Superior, CAPES; do Conselho Nacional de
Desenvolvimento Científico e Tecnológico, CNPq; do Instituto Nacional de Ciência
e Tecnologia de Astrofísica, INCT-A; do programa Ciência sem Fronteiras; e do
Instituto de Astrofísica de Andalucía, IAA/CSIC. Também sou grato ao Curso de
Pós Graduação em Física, tanto pelo apoio financeiro quanto pela estrutura e
instalações adequadas.

Finalmente, agradeço à colaboração do CALIFA, pelas inúmeras discussões e
sugestões, especialmente nas {\em Busy Weeks}, sempre produtivas. E
principalmente à equipe de observação e redução por produzir os dados com tanto
capricho e dedicação, fundações sobre as quais este trabalho se ergueu.

\clearpage

% End of Acknowledgments
