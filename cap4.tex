%%%%%%%%%%%%%%%%%%%%%%%%%%%%%%%%%%%%%%%%%%%%%%%%%%%%%%%%%%%%%%%%%
% Tese de Doutorado / Dept Fisica, CFM, UFSC                    %
% Andre@UFSC - 2014                                             %
%%%%%%%%%%%%%%%%%%%%%%%%%%%%%%%%%%%%%%%%%%%%%%%%%%%%%%%%%%%%%%%%%

%:::::::::::::::::::::::::::::::::::::::::::::::::::::::::::::::%
%                                                               %
%                          Capítulo 4                           %
%                                                               %
%:::::::::::::::::::::::::::::::::::::::::::::::::::::::::::::::%

%***************************************************************%
%                                                               %
%                   Morfologia de galáxias                      %
%                                                               %
%***************************************************************%

\chapter{Morfologia de galáxias}
\label{sec:morph}

%***************************************************************%
%                                                               %
%                         Section 1                             %
%                                                               %
%***************************************************************%

\section{Sequência de Hubble}

\TODO Classificação visual, sequencia de Hubble. Caracteristicas de galáxias:
elipticas, espirais, etc.

\section{Outras formas de classificação}

Índice de concentração.

Lei de deVauculeurs.

Exponencial.

Sérsic.

\section{Separação em componentes morfológicas}

O que são bojos, discos, halos, barras, etc.

Motivação científica da separação disco e bojo.


% End of this chapter
