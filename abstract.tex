%%%%%%%%%%%%%%%%%%%%%%%%%%%%%%%%%%%%%%%%%%%%%%%%%%%%%%%%%%%%%%%%%
% Dissertacao de Mestrado / Dept Fisica, CFM, UFSC              %
% Andre@UFSC - 2011                                             %
%%%%%%%%%%%%%%%%%%%%%%%%%%%%%%%%%%%%%%%%%%%%%%%%%%%%%%%%%%%%%%%%%

%***************************************************************%
%                                                               %
%                          Abstract                             %
%                                                               %
%***************************************************************%

\begin{abstract}

In this work we have developed some tools to work with Integral Field
Spectroscopy (IFS) spectra from the CALIFA survey. The spectra from the spaxels are
preprocessed, and then analyzed using the software \starlight. One of the main
tools discussed here, PyCASSO, organizes the output from \starlight into
datacubes of stellar population synthesis products. It also allows for
interactive exploratory programming, giving easy access to the multi-dimensional
data.

Using these tools, we developed a method for recovering and analysing the
stellar populations of the different morphological components of galaxies using
IFS data. Using the software Imfit, wrapped in Python, we perform the
decomposition of a sample of 43 candidate S0 galaxies from the CALIFA Survey
into a Sérsic bulge and an exponential disk. The decomposition is made
wavelength-wise, so that at the end we get the bulge and disk spectra for each
pixel. A good PSF measurement is critical to this process, so we perform a
characterization of the PSF using the calibration stars from the survey. The
morphological parameters ($r_e$, $n$, P.A. and $\epsilon$ for bulge, $h$, P.A.
and $\epsilon$ for disk) in most cases depend linearly on the wavelength, on
average, but their behavior at each $\lambda$ are not yet fully understood.
Using the decomposed spectra from the 9 best decompositions from the sample we
apply a stellar population synthesis using \starlight, in order to recover the
stellar populations of bulge and disk. Only two of those galaxies produced
\starlight fits with a small residual, CALIFA 0592 (NGC 4874) and CALIFA 0858
(UGC 10905).
In both cases we recover an older and lower-metallicity bulge and a younger and
higher-metallicity disk (in comparison to the whole galaxy). The stellar
synthesis produces more robust results using the integrated spectra. The
spatially resolved spectra from bulge and disk seem to have artifacts that get
interpreted as dust, among other things, thus leading to wrong results.

% TODO: Develop this argument (about spatially resolved spectra) a bit more?


\end{abstract}

% End of abstract



